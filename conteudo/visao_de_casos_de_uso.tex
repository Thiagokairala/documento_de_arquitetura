A parte de maior risco arquitetural é a parte de acesso ao \textit{web service} então esta sessão descreverá como será a interação da camada controller com este módulod o sistema

\subsection{Realizações de Casos de Uso}
	
	\subsubsection{Pesquisar um parlamentar}

		\begin{enumerate}
			\item Ao receber a requisição a camada da \textit{view} irá pedir a \textit{controller} para que a mesma faça as contas da estatística;

			\item A \textit{controller} irá pedir os dados para a camada de \textit{dataParser} para que consiga os dados e entregue-os em um formato esperado;

<<<<<<< HEAD
			\item A classe parser irá tentar buscar os dados do \textit{web service} pelas classes presentes no pacote \textit{webServiceConnector}, caso funcione irá retornar os dados para a controller do jeito que a mesma espera;
=======
			\item A classe parser irá tentar buscar os dados do \textit{web service}, caso funcione irá retornar os dados para a controller do jeito que a mesma espera;
>>>>>>> f7c0dd0... Finalizada visao de casos de uso

			\item Se por algum motivo a classe parser tiver alguma dificuldade em recuperar estas informações, será feita então uma requisição a camada \textit{dao} pelas informações no banco de dados;

			\item Tendo os dados em mão o sistema irá calcular as estatísticas e finalmente enviá-los para a \textit{view} para que possam ser mostrados ao usuário.
		\end{enumerate}