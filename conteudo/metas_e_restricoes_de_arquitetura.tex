%[Esta seção descreve os requisitos de software e os objetivos que têm um impacto significativo na arquitetura, como proteção, segurança, privacidade, uso de um produto desenvolvido internamente e adquirido pronto para ser usado, portabilidade, distribuição e reutilização. Ela também captura as restrições especiais que podem ser aplicáveis: estratégia de design e implementação, ferramentas de desenvolvimento, estrutura das equipes, cronograma, código-fonte legado e assim por diante.]
Durante o desenvolvimento da arquitetura foram levados em conta diversos aspectos como por exemplo a segurança do sistema, a possibilidade de diversas verificações para garantir sucesso e evitar o famoso \textit{"garbage in, garbage out"}, a possibilidade de o sistema vir a crescer e por último possibilitar pessoas externas a se identificarem com o sistema e contribuir com novas funcionalidades.

A maior preocupação existente na arquitetura projetada foi a formatação dos dados vindos do web service, pois existe a possibilidade de haver a mudança e não é de controle da equipe de desenvolvimento, por isso fazer um sistema modularizado para facilitar a troca da entrada de dados caso seja necessário.