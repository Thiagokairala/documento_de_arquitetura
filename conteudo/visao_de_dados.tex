%[Uma descrição da perspectiva de armazenamento de dados persistentes do sistema. Esta seção será opcional se os dados persistentes forem poucos ou inexistentes ou se a conversão entre o Modelo de Design e o Modelo de Dados for trivial.]

O pacote \textit{Dao}, que fará toda a conexão com o banco de dados irá obter dados dos Deputados e das Sessões. Com estes dados, será possível conclusão de todas as funcionalidades propostas pelo \textit{Chamada Parlamentar}.

\subsection{Deputados:}
	
	Seguem todos os dados relacionados aos Deputados da Câmara dos Deputados que serão armazenados no sistema:

	\begin{itemize}
		\item \textbf{ID:}

			O campo ID será utilizado para armazenar o registro de ID do deputado, para facilitar a procura pelo mesmo.

		\item \textbf{civilName:}

			O campo civilName será utilizado para armazenar o nome civil do deputado, ou seja, o seu nome oficial.

		\item \textbf{email:}

			O campo email será utilizado para armazenar o email do deputado.

		\item \textbf{gender:}

			O campo gender será utilizado para armazenar o gênero do deputado (masculino ou feminino).

		\item \textbf{idParliamentary:}

			O campo idParliamentary será utilizado para armazenar o id de parlamentar do deputado. A diferença deste para o ID é que este é definido pela Câmara dos Deputados, este foi criado para organização no Banco de Dados.

		\item \textbf{nascimento:}

			O campo nascimento será utilizado para armazenar a data de nascimento do deputado.

		\item \textbf{officeBuilding:}

			O campo officeBuilding será utilizado para armazenar o anexo onde o deputado trabalha.

		\item \textbf{officeNumber:}

			O campo officeNumber será utilizado para armazenar o número do escritório do deputado.

		\item \textbf{phone:}

			O campo phone será utilizado para armazenar o telefone do deputado.

		\item \textbf{politicalParty:}

			O campo politicalParty será utilizado para armazenar o partido político do deputado.

		\item \textbf{treatmentName:}

			O campo treatmentName será utilizado para armazenar o nome de tratamento do deputado.

		\item \textbf{uf:}

			O campo uf será utilizado para armazenar a unidade federativa onde o deputado foi eleito.

			

			
			
			

			
			

			

			

			

	\end{itemize}
ADICIONAR O MER AQUI E EXPLICÁ-lo